\documentclass[float=false, crop=false]{standalone}
\begin{document}

\section{Introduction}

This is going to be a different kind of session than the others in our series on
parsing and compiler construction. For this session, there won't be any
experiments to do or code to write. This once, I'd like to just talk with you
for a while. Mercifully, it will be a short session, and then we can take up
where we left off, hopefully with renewed vigor.

When I was in college, I found that I could always follow a prof's lecture a lot
better if I knew where he was going with it. I'll bet you were the same.

So I thought maybe it's about time I told you where we're going with this
series: what's coming up in future installments, and in general what all this is
about. I'll also share some general thoughts concerning the usefulness of what
we've been doing.

\section{The Road Home}

So far, we've covered the parsing and translation of arithmetic expressions,
Boolean expressions, and combinations connected by relational operators. We've
also done the same for control constructs. In all of this we've leaned heavily
on the use of top-down, recursive descent parsing, BNF definitions of the
syntax, and direct generation of assembly-language code. We also learned the
value of such tricks as single-character tokens to help us see the forest
through the trees. In the last installment we dealt with lexical scanning, and I
showed you simple but powerful ways to remove the single-character barriers.

Throughout the whole study, I've emphasized the KISS philosophy ... Keep It
Simple, Sidney ... and I hope by now you've realized just how simple this stuff
can really be. While there are for sure areas of compiler theory that are truly
intimidating, the ultimate message of this series is that in practice you can
just politely sidestep many of these areas. If the language definition
cooperates or, as in this series, if you can define the language as you go, it's
possible to write down the language definition in BNF with reasonable ease. And,
as we've seen, you can crank out parse procedures from the BNF just about as
fast as you can type.

As our compiler has taken form, it's gotten more parts, but each part is quite
small and simple, and very much like all the others.

At this point, we have many of the makings of a real, practical compiler. As a
matter of fact, we already have all we need to build a toy compiler for a
language as powerful as, say, Tiny BASIC. In the next couple of installments,
we'll go ahead and define that language.

To round out the series, we still have a few items to cover. These include:

\begin{emeration}
  \item Procedure calls, with and without parameters
  \item Local and global variables
  \item Basic types, such as character and integer types
  \item Arrays
  \item Strings
  \item User-defined types and structures
  \item Tree-structured parsers and intermediate languages
  \item Optimization
\end{enumeration}

These will all be covered in future installments. When we're finished, you'll
have all the tools you need to design and build your own languages, and the
compilers to translate them.

I can't design those languages for you, but I can make some comments and
recommendations. I've already sprinkled some throughout past installments.
You've seen, for example, the control constructs I prefer.

These constructs are going to be part of the languages I build. I have three
languages in mind at this point, two of which you will see in installments to
come:

TINY - A  minimal,  but  usable  language  on the order  of  Tiny
       BASIC or Tiny C.  It won't be very practical, but  it will
       have enough power to let you write and  run  real programs
       that do something worthwhile.

KISS - The  language  I'm  building for my  own  use.    KISS  is
       intended to be  a  systems programming language.  It won't
       have strong typing  or  fancy data structures, but it will
       support most of  the  things  I  want to do with a higher-
       order language (HOL), except perhaps writing compilers.

I've also been toying for years with the idea of a HOL-like assembler, with
structured control constructs and HOL-like assignment statements. That, in fact,
was the impetus behind my original foray into the jungles of compiler theory.
This one may never be built, simply because I've learned that it's actually
easier to implement a language like KISS, that only uses a subset of the CPU
instructions. As you know, assembly language can be bizarre and irregular in the
extreme, and a language that maps one-for-one onto it can be a real challenge.
Still, I've always felt that the syntax used in conventional assemblers is dumb
... why is

     MOVE.L A,B

better, or easier to translate, than

     B=A ?

I think it would be an interesting exercise to develop a "compiler" that would
give the programmer complete access to and control over the full complement of
the CPU instruction set, and would allow you to generate programs as efficient
as assembly language, without the pain of learning a set of mnemonics. Can it be
done? I don't know. The real question may be, "Will the resulting language be
any easier to write than assembly"? If not, there's no point in it. I think that
it can be done, but I'm not completely sure yet how the syntax should look.

Perhaps you have some comments or suggestions on this one. I'd love to hear
them.

You probably won't be surprised to learn that I've already worked ahead in most
of the areas that we will cover. I have some good news: Things never get much
harder than they've been so far. It's possible to build a complete, working
compiler for a real language, using nothing but the same kinds of techniques
you've learned so far. And THAT brings up some interesting questions.


\section{Why Is It So Simple?}

Before embarking on this series, I always thought that compilers were just
naturally complex computer programs ... the ultimate challenge. Yet the things
we have done here have usually turned out to be quite simple, sometimes even
trivial.

For awhile, I thought is was simply because I hadn't yet gotten into the meat of
the subject. I had only covered the simple parts. I will freely admit to you
that, even when I began the series, I wasn't sure how far we would be able to go
before things got too complex to deal with in the ways we have so far. But at
this point I've already been down the road far enough to see the end of it.
Guess what?


                     THERE ARE NO HARD PARTS!


Then, I thought maybe it was because we were not generating very good object
code. Those of you who have been following the series and trying sample compiles
know that, while the code works and is rather foolproof, its efficiency is
pretty awful. I figured that if we were concentrating on turning out tight code,
we would soon find all that missing complexity.

To some extent, that one is true. In particular, my first few efforts at trying
to improve efficiency introduced complexity at an alarming rate. But since then
I've been tinkering around with some simple optimizations and I've found some
that result in very respectable code quality, WITHOUT adding a lot of
complexity.

Finally, I thought that perhaps the saving grace was the "toy compiler" nature
of the study. I have made no pretense that we were ever going to be able to
build a compiler to compete with Borland and Microsoft. And yet, again, as I get
deeper into this thing the differences are starting to fade away.

Just to make sure you get the message here, let me state it flat out:

   USING THE TECHNIQUES WE'VE USED  HERE,  IT  IS  POSSIBLE TO
   BUILD A PRODUCTION-QUALITY, WORKING COMPILER WITHOUT ADDING
   A LOT OF COMPLEXITY TO WHAT WE'VE ALREADY DONE.


Since the series began I've received some comments from you. Most of them echo
my own thoughts: "This is easy! Why do the textbooks make it seem so hard?" Good
question.

Recently, I've gone back and looked at some of those texts again, and even
bought and read some new ones. Each time, I come away with the same feeling:
These guys have made it seem too hard.

What's going on here? Why does the whole thing seem difficult in the texts, but
easy to us? Are we that much smarter than Aho, Ullman, Brinch Hansen, and all
the rest?

Hardly. But we are doing some things differently, and more and more I'm starting
to appreciate the value of our approach, and the way that it simplifies things.
Aside from the obvious shortcuts that I outlined in Part I, like
single-character tokens and console I/O, we have made some implicit assumptions
and done some things differently from those who have designed compilers in the
past. As it turns out, our approach makes life a lot easier.

So why didn't all those other guys use it?

You have to remember the context of some of the earlier compiler development.
These people were working with very small computers of limited capacity. Memory
was very limited, the CPU instruction set was minimal, and programs ran in batch
mode rather than interactively. As it turns out, these caused some key design
decisions that have really complicated the designs. Until recently, I hadn't
realized how much of classical compiler design was driven by the available
hardware.

Even in cases where these limitations no longer apply, people have tended to
structure their programs in the same way, since that is the way they were taught
to do it.

In our case, we have started with a blank sheet of paper. There is a danger
there, of course, that you will end up falling into traps that other people have
long since learned to avoid. But it also has allowed us to take different
approaches that, partly by design and partly by pure dumb luck, have allowed us
to gain simplicity.

Here are the areas that I think have led to complexity in the past:

  o  Limited RAM Forcing Multiple Passes

     I  just  read  "Brinch  Hansen  on  Pascal   Compilers"  (an
     excellent book, BTW).  He  developed a Pascal compiler for a
     PC, but he started the effort in 1981 with a 64K system, and
     so almost every design decision  he made was aimed at making
     the compiler fit  into  RAM.    To do this, his compiler has
     three passes, one of which is the lexical scanner.  There is
     no way he could, for  example, use the distributed scanner I
     introduced  in  the last installment,  because  the  program
     structure wouldn't allow it.  He also required  not  one but
     two intermediate  languages,  to  provide  the communication
     between phases.

     All the early compiler writers  had to deal with this issue:
     Break the compiler up into enough parts so that it  will fit
     in memory.  When  you  have multiple passes, you need to add
     data structures to support the  information  that  each pass
     leaves behind for the next.   That adds complexity, and ends
     up driving the  design.    Lee's  book,  "The  Anatomy  of a
     Compiler,"  mentions a FORTRAN compiler developed for an IBM
     1401.  It had no fewer than 63 separate passes!  Needless to
     say,  in a compiler like this  the  separation  into  phases
     would dominate the design.

     Even in  situations  where  RAM  is  plentiful,  people have
     tended  to  use  the same techniques because  that  is  what
     they're familiar with.   It  wasn't  until Turbo Pascal came
     along that we found how simple a compiler could  be  if  you
     started with different assumptions.


  o  Batch Processing

     In the early days, batch  processing was the only choice ...
     there was no interactive computing.   Even  today, compilers
     run in essentially batch mode.

     In a mainframe compiler as  well  as  many  micro compilers,
     considerable effort is expended on error recovery ... it can
     consume as much as 30-40%  of  the  compiler  and completely
     drive the design.  The idea is to avoid halting on the first
     error, but rather to keep going at all costs,  so  that  you
     can  tell  the  programmer about as many errors in the whole
     program as possible.

     All of that harks back to the days of the  early mainframes,
     where turnaround time was measured  in hours or days, and it
     was important to squeeze every last ounce of information out
     of each run.

     In this series, I've been very careful to avoid the issue of
     error recovery, and instead our compiler  simply  halts with
     an error message on  the  first error.  I will frankly admit
     that it was mostly because I wanted to take the easy way out
     and keep things simple.   But  this  approach,  pioneered by
     Borland in Turbo Pascal, also has a lot going for it anyway.
     Aside from keeping the  compiler  simple,  it also fits very
     well  with   the  idea  of  an  interactive  system.    When
     compilation is  fast, and especially when you have an editor
     such as Borland's that  will  take you right to the point of
     the error, then it makes a  lot  of sense to stop there, and
     just restart the compilation after the error is fixed.


  o  Large Programs

     Early compilers were designed to handle  large  programs ...
     essentially infinite ones.    In those days there was little
     choice;  the  idea  of  subroutine  libraries  and  separate
     compilation  were  still  in  the  future.      Again,  this
     assumption led to  multi-pass designs and intermediate files
     to hold the results of partial processing.

     Brinch Hansen's  stated goal was that the compiler should be
     able to compile itself.   Again, because of his limited RAM,
     this drove him to a multi-pass design.  He needed  as little
     resident compiler code as possible,  so  that  the necessary
     tables and other data structures would fit into RAM.

     I haven't stated this one yet, because there  hasn't  been a
     need  ... we've always just read and  written  the  data  as
     streams, anyway.  But  for  the  record,  my plan has always
     been that, in  a  production compiler, the source and object
     data should all coexist  in  RAM with the compiler, a la the
     early Turbo Pascals.  That's why I've been  careful  to keep
     routines like GetChar  and  Emit  as  separate  routines, in
     spite of their small size.   It  will be easy to change them
     to read to and write from memory.


  o  Emphasis on Efficiency

     John  Backus has stated that, when  he  and  his  colleagues
     developed the original FORTRAN compiler, they KNEW that they
     had to make it produce tight code.  In those days, there was
     a strong sentiment against HOLs  and  in  favor  of assembly
     language, and  efficiency was the reason.  If FORTRAN didn't
     produce very good  code  by  assembly  standards,  the users
     would simply refuse to use it.  For the record, that FORTRAN
     compiler turned out to  be  one  of  the most efficient ever
     built, in terms of code quality.  But it WAS complex!

     Today,  we have CPU power and RAM size  to  spare,  so  code
     efficiency is not  so  much  of  an  issue.    By studiously
     ignoring this issue, we  have  indeed  been  able to Keep It
     Simple.    Ironically,  though, as I have said, I have found
     some optimizations that we can  add  to  the  basic compiler
     structure, without having to add a lot of complexity.  So in
     this  case we get to have our cake and eat it too:  we  will
     end up with reasonable code quality, anyway.


  o  Limited Instruction Sets

     The early computers had primitive instruction sets.   Things
     that  we  take  for granted, such as  stack  operations  and
     indirect addressing, came only with great difficulty.

     Example: In most compiler designs, there is a data structure
     called the literal pool.  The compiler  typically identifies
     all literals used in the program, and collects  them  into a
     single data structure.    All references to the literals are
     done  indirectly  to  this  pool.    At  the   end   of  the
     compilation, the  compiler  issues  commands  to  set  aside
     storage and initialize the literal pool.

     We haven't had to address that  issue  at all.  When we want
     to load a literal, we just do it, in line, as in

          MOVE #3,D0

     There is something to be said for the use of a literal pool,
     particularly on a machine like  the 8086 where data and code
     can  be separated.  Still, the whole  thing  adds  a  fairly
     large amount of complexity with little in return.

     Of course, without the stack we would be lost.  In  a micro,
     both  subroutine calls and temporary storage depend  heavily
     on the stack, and  we  have used it even more than necessary
     to ease expression parsing.


  o  Desire for Generality

     Much of the content of the typical compiler text is taken up
     with issues we haven't addressed here at all ... things like
     automated  translation  of  grammars,  or generation of LALR
     parse tables.  This is not simply because  the  authors want
     to impress you.  There are good, practical  reasons  why the
     subjects are there.

     We have been concentrating on the use of a recursive-descent
     parser to parse a  deterministic  grammar,  i.e.,  a grammar
     that is not ambiguous and, therefore, can be parsed with one
     level of lookahead.  I haven't made much of this limitation,
     but  the  fact  is  that  this represents a small subset  of
     possible grammars.  In fact,  there is an infinite number of
     grammars that we can't parse using our techniques.    The LR
     technique is a more powerful one, and can deal with grammars
     that we can't.

     In compiler theory, it's important  to know how to deal with
     these  other  grammars,  and  how  to  transform  them  into
     grammars  that  are  easier to deal with.  For example, many
     (but not all) ambiguous  grammars  can  be  transformed into
     unambiguous ones.  The way to do this is not always obvious,
     though, and so many people  have  devoted  years  to develop
     ways to transform them automatically.

     In practice, these  issues  turn out to be considerably less
     important.  Modern languages tend  to be designed to be easy
     to parse, anyway.   That  was a key motivation in the design
     of Pascal.   Sure,  there are pathological grammars that you
     would be hard pressed to write unambiguous BNF  for,  but in
     the  real  world  the best answer is probably to avoid those
     grammars!

     In  our  case,  of course, we have sneakily let the language
     evolve  as  we  go, so we haven't painted ourselves into any
     corners here.  You may not always have that luxury.   Still,
     with a little  care  you  should  be able to keep the parser
     simple without having to resort to automatic  translation of
     the grammar.


We have taken a vastly different approach in this series. We started with a
clean sheet of paper, and developed techniques that work in the context that we
are in; that is, a single-user PC with rather ample CPU power and RAM space. We
have limited ourselves to reasonable grammars that are easy to parse, we have
used the instruction set of the CPU to advantage, and we have not concerned
ourselves with efficiency. THAT's why it's been easy.

Does this mean that we are forever doomed to be able to build only toy
compilers? No, I don't think so. As I've said, we can add certain optimizations
without changing the compiler structure. If we want to process large files, we
can always add file buffering to do that. These things do not affect the overall
program design.

And I think that's a key factor. By starting with small and limited cases, we
have been able to concentrate on a structure for the compiler that is natural
for the job. Since the structure naturally fits the job, it is almost bound to
be simple and transparent. Adding capability doesn't have to change that basic
structure. We can simply expand things like the file structure or add an
optimization layer. I guess my feeling is that, back when resources were tight,
the structures people ended up with were artificially warped to make them work
under those conditions, and weren't optimum structures for the problem at hand.

\section{Conclusion}

Anyway, that's my arm-waving guess as to how we've been able to keep things
simple. We started with something simple and let it evolve naturally, without
trying to force it into some traditional mold.

We're going to press on with this. I've given you a list of the areas we'll be
covering in future installments. With those installments, you should be able to
build complete, working compilers for just about any occasion, and build them
simply. If you REALLY want to build production-quality compilers, you'll be able
to do that, too.

For those of you who are chafing at the bit for more parser code, I apologize
for this digression. I just thought you'd like to have things put into
perspective a bit. Next time, we'll get back to the mainstream of the tutorial.

So far, we've only looked at pieces of compilers, and while we have many of the
makings of a complete language, we haven't talked about how to put it all
together. That will be the subject of our next two installments. Then we'll
press on into the new subjects I listed at the beginning of this installment.

See you then.

\end{document}