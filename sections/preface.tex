\documentclass[float=false, crop=false]{standalone}
\begin{document}

This file contains all of the installments of Jack Crenshaw's tutorial on
compiler construction, including the new Installment 15. The intended audience
is those folks who are not computer scientists, but who enjoy computing and have
always wanted to know how compilers work. A lot of compiler theory has been left
out, but the practical issues are covered. By the time you have completed the
series, you should be able to design and build your own working compiler. It
will not be the world's best, nor will it put out incredibly tight code. Your
product will probably never put Borland or MicroSoft out of business. But it
will work, and it will be yours.

A word about the file format: The files were originally created using Borland's
DOS editor, Sprint. Sprint could write to a text file only if you formatted the
file to go to the selected printer. I used the most common printer I could think
of, the Epson MX-80, but even then the files ended up with printer control
sequences at the beginning and end of each page.

To bring the files up to date and get myself positioned to continue the series,
I recently (1994) converted all the files to work with Microsoft Word for
Windows. Unlike Sprint, Word allows you to write the file as a DOS text file.
Unfortunately, this gave me a new problem, because when Word is writing to a
text file, it doesn't write hard page breaks or page numbers. In other words, in
six years we've gone from a file with page breaks and page numbers, but embedded
escape sequences, to files with no embedded escape sequences but no page breaks
or page numbers. Isn't progress wonderful?

Of course, it's possible for me to insert the page numbers as straight text,
rather than asking the editor to do it for me. But since Word won't allow me to
write page breaks to the file, we would end up with files with page numbers that
may or may not fall at the ends of the pages, depending on your editor and your
printer. It seems to me that almost every file I've ever downloaded from
CompuServe or BBS's that had such page numbering was incompatible with my
printer, and gave me pages that were one line short or one line long, with the
page numbers consequently walking up the page.

So perhaps this new format is, after all, the safest one for general
distribution. The files as they exist will look just fine if read into any text
editor capable of reading DOS text files. Since most editors these days include
rather sophisticated word processing capabilities, you should be able to get
your editor to paginate for you, prior to printing.

I hope you like the tutorials. Much thought went into them.


									Jack W. Crenshaw
								CompuServe 72325,1327
\end{document}